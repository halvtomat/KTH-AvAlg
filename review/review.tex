\documentclass[11pt, a4paper]{article}

\title{Chapter 3.1 Review}
\author{Daniel Gustafsson}
\date{December 2021}

\begin{document}
\maketitle

\section{Vertex Cover}
\subsection{Problem definition}
I had to look up the problem definition on Wikipedia to understand what the problem was.
"A vertex cover is a set of vertices that "covers" all edges." This sentence means nothing to me,
I would need an explanation of the meaning of "covers" in this context to understand the problem.

\subsection{Algorithm 1}
Algorithm description is OK. There is no explanation provided for the approximation ratio. 

The question below it is confusing to me, how can a vertex cover
have a size k? Does this mean that the cover should include exactly k nodes or k edges? Something else?

\subsection{Algorithm 2}
No complaints here except that I still don't know what k represents.

\subsection{Algorithm 3}
At this point I think k is some limit on how many nodes can be in S.
The algorithm description is understandable.

The example run drawn during lecture is just a graph, means nothing to me.

The images in the Analysis section is very hard to understand.

In the third point of the time complexity section it should be "nodes" or "node(s)" instead of just "node".
Apart form that the analysis is OK.

The first time OPT is defined is in the "Correctness of ALG3" section, the definition should be at the first
occurrence.

\subsection{Parameterized Problems}
I'm not familiar with 3-SAT so the definition is a bit hard to understand.

\section{Kernelization}
\subsection{Motivation}
Please recreate the image digitally for better clarity.

\subsection{Heuristic}
Please recreate all images digitally for better clarity.

\subsection{Algorithm 4}
Description and pseudo code easy to understand.

\end{document}