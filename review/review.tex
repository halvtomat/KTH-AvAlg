\documentclass[11pt, a4paper]{article}

\title{Chapter 3.1 Review}
\author{Daniel Gustafsson}
\date{December 2021}

\begin{document}
\maketitle

\section{Vertex Cover}
\subsection{Problem definition}
I had to look up the problem definition on Wikipedia to understand what the problem was.
"A vertex cover is a set of vertices that "covers" all edges." This sentence means nothing to me,
I would need an explanation of the meaning of "covers" in this context to understand the problem.
I think an example picture would suffice to properly explain the problem.
\newline
\textbf{Note:} The explanation seems to be sufficient for me NOW, after looking it up, but at first
look it didn't make sense.

\subsection{Algorithm 1}
Algorithm description is OK. There is no explanation provided for the approximation ratio. 
\newline
The question below it is confusing to me, how can a vertex cover
have a size k? Does this mean that the cover should include exactly k nodes or k edges? Something else?
\newline
\textbf{Note:} After explanation from a TA I understand what k is but k should still be better explained in the slides
to make life easier for everyone.

\subsection{Algorithm 2}
No complaints here except that I still don't know what k represents.

\subsection{Algorithm 3}
At this point I think k is some limit on how many nodes can be in S.
The algorithm description is understandable but the case where both recursions returns a set S
is not covered, I assume some check is made to see which is the smaller set and then return that one.
\newline
The example run drawn during lecture is just a graph, means nothing to me.
\newline
The hand drawn images in the Analysis section is very hard to read.
\newline
In the third point of the time complexity section it should be "nodes" or "node(s)" instead of just "node".
Apart form that the analysis is OK and quite easy to understand.
\newline
The first time OPT is defined is in the "Correctness of ALG3" section, the definition should be at the first
occurrence.

\subsection{Parameterized Problems}
I'm not familiar with 3-SAT so that example is a bit hard to understand.
\newline
This is otherwise quite easy to understand as there are so many examples.
Same with the FPT slide.

\section{Kernelization}
\subsection{Motivation}
Please recreate the image digitally for better clarity.
Apart from the image, the description is easy to understand.

\subsection{Heuristic}
Please recreate all images digitally for better clarity.
I was not previously aware that the degree of a node is how many edges connects to it, this should probably
be explained in the slide.
Apart from these problems the description is nice.

\subsection{Algorithm 4}
Step 4 is not obvious to me, please provide an explanation as to why $OPT(G)>k$ should be returned here.
The analysis is overall pretty cluttered, should be spread out over at least 2 slides instead of the one and
use the same font size as the rest of the slides instead of the smaller variant.
Because of the smaller font and big block of text this analysis feels quite overwhelming.

\section{Final notes}
Overall I understood the presentation much better at my second pass of reading through it.
Seems to me that most of the presentation is made to work in a classroom and not to be read
at home. My biggest issues is the vague definitions of some variables like k, OPT and "cover"
and that all images are hand drawn instead of digitally created.
\newline
I think that if these issues are addressed the presentation would be much easier to understand for everyone.

\end{document}