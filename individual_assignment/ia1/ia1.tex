\documentclass[11pt, a4paper]{article}

\usepackage[english]{babel}
\usepackage{listings}

\title{Individual Assignment 1}
\author{Daniel Gustafsson}

\begin{document}
\maketitle	

\section{Algorithm}
\subsection{Description}
\noindent\textit{order} = ordered list of gene markers.

\noindent\textit{ratio} = approximation ratio variable.

\vspace{2mm}
To calculate the approximation ratio, check each observation
for compatibility with \textit{order}.

For each compatible observation, add one to \textit{ratio}.

When all observations have been accounted for, divide \textit{ratio}
by the number of observations.

Shuffle \textit{order} until an approximation
ratio of at least 0.2 is reached.

\subsection{Pseudo code}
\lstinputlisting[language=Python, firstline=3]{ia1.py}

\subsection{Proof of correctness}
For a given ordered list of genes $(x_0, x_1, ... , x_n)$ each
observation is a set of three genes $(x_i, x_j, x_k)$.

For an observation to be compatible with the order of the list,
the elements in the observation must be in some specific order
in the list.

The list can only order the observed elements in one of
the following six ways:

\begin{enumerate}
\item $(x_i, x_j, x_k)$

\item $(x_i, x_k, x_j)$

\item $(x_j, x_i, x_k)$

\item $(x_j, x_k, x_i)$

\item $(x_k, x_i, x_j)$

\item $(x_k, x_j, x_i)$
\end{enumerate}

The observation is compatible with orders 1 and 6, or 
$1/3$ out of the available orders. 

This means that the probability that any observation is
compatible with any order is $1/3$.

\[Pr[ratio \ge 0.2] = 1 - (1/3)^m\]

When re-shuffling the order \textit{k} times, 

\[Pr[ratio \ge 0.2] = 1 - \prod_{1}^{k} Pr[ratio < 0.2] = 1\]

given a great enough \textit{k}.

If choosing $k = 100$, $Pr[ratio < 0.2] = 2.5*10^{-18}$.

\subsection{Time complexity analysis}

Shuffling a list is $O(n)$, 
calculating the \textit{ratio} is $O(m)$ and
these operations are executed \textit{k} times where \textit{k} is some constant.

\vspace{1mm}
The resulting time complexity is $O(n+m)$.

\end{document}