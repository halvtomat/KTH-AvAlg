\documentclass[11pt, a4paper]{article}

\usepackage[english]{babel}
\usepackage{listings}

\title{Individual Assignment 1}
\author{Daniel Gustafsson}

\begin{document}
\maketitle	

\section{Algorithm}
\subsection{Description}
\noindent\textit{order} = ordered list of gene markers.

\noindent\textit{ratio} = approximation ratio variable.

\noindent To calculate the approximation ratio, check each observation
for compatibility with \textit{order}.

\noindent For each compatible observation, add one to \textit{ratio}.

\noindent When all observations have been accounted for, divide \textit{ratio}
by the number of observations.

\noindent Shuffle \textit{order} until an approximation
ratio of at least 0.2 is reached.

\subsection{Pseudo code}
\lstinputlisting[language=Python, firstline=3]{ia1.py}

\subsection{Proof of correctness}
For a given ordered list of genes $(x_0, x_1, ... , x_n)$ each
observation is a set of three genes $(x_i, x_j, x_k)$.

\noindent For an observation to be compatible with the order of the list,
the elements in the observation must be in some specific order
in the list.

\noindent The list can only order the observed elements in one of
the following six ways:

\begin{enumerate}
\item $(x_i, x_j, x_k)$

\item $(x_i, x_k, x_j)$

\item $(x_j, x_i, x_k)$

\item $(x_j, x_k, x_i)$

\item $(x_k, x_i, x_j)$

\item $(x_k, x_j, x_i)$
\end{enumerate}

\noindent The observation is compatible with orders 1 and 6, or 
$1/3$ out of the available orders. 

\noindent This means that the probability that any observation is
compatible with any order is $1/3$.

\noindent When re-shuffling the order \textit{k} times, $Pr[ratio \ge 0.2] = 1 - \prod_{1}^{k} Pr[ratio < 0.2] = 1$ given a great enough \textit{k}

\noindent When \textit{m} is bigger, $Pr[ratio > 0.2]$ is bigger,
the worst case is when $\textit{m} = 1$, then $Pr[ratio \ge 0.2] = Pr[ratio = 1.0] = 1/3$.

\end{document}