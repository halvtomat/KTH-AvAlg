\documentclass[11pt, a4paper]{article}

\usepackage{amsmath}
\usepackage{algorithm}
\usepackage{algpseudocode}

\newtheorem{theorem}{Theorem}[section]
\newtheorem{lemma}[theorem]{Lemma}

\title{Individual Assignment 2.5 (Grade E)}
\author{Daniel Gustafsson}
\date{December 2021}

\begin{document}
\maketitle

\section{Solution}
The suggested solution is to use a variant of the Misra-Gries algorithm to calculate the
frequency of the 100 most common elements in a data stream and then check if the most
frequent element out of those appears more than 50\% of the time in the data stream.

If some element appears more than 50\% of the time, return that element, otherwise return
"No majority".

\subsection{Pseudo code}

\begin{algorithmic}[1]
\State $dictionary \gets \emptyset$
\State $m \gets 0$
\For{$a_i$ in $a_1,...,a_m$}
	\State $m \gets m + 1$
	\If{$dictionary$ contains $a_i$}
		\State $dictionary[a_i] \gets dictionary[a_i] + 1$ \Comment{Increment item}
	\ElsIf{lenght of $dictionary < 99$}
		\State $dictionary \gets dictionary \cup (a_i, 1)$ \Comment{Add to dictionary}
	\Else
		\For{$k$ in $dictionary$} \Comment{Decrement all items}
			\State $dictionary[k] \gets dictionary[k] - 1$
			\If{$dictionary[k] = 0$}
				\State $dictionary \gets dictionary \setminus dictionary[k]$
			\EndIf
		\EndFor
	\EndIf
\EndFor
\For{$k$ in $dictionary$}
	\If{$dictionary[k] > m/2$}
		\State \Return $k$ \Comment{$k$ is the majority element}
	\EndIf
\EndFor
\State \Return "No majority"
\end{algorithmic}

\section{Proof}

\vspace{2mm}\noindent\fbox{\parbox{\textwidth}{\textbf{Lemma 1:} The Misra-Gries algorithm will find
all elements occuring more than $m/k$ times.
\newline

\textit{Proof}: Proven in class.

}}

\vspace{2mm}\noindent\fbox{\parbox{\textwidth}{\textbf{Lemma 2:} If a majority element exists,
it will be in the dictionary.
\newline

\textit{Proof}: Since $k = 100$, all elements occuring more than $m/100$ times or 
\textit{more than 1\% of times} will be in the dictionary according to Lemma 1. Since $50\%$ is more than
$1\%$, the majority element will be in the dictionary if it exists.

}}

\vspace{2mm}\noindent\fbox{\parbox{\textwidth}{\textbf{Lemma 3:} The count value in the dictionary
will in the span $f - m/k \le count \le f$ where $f$ is the true frequency.
\newline

\textit{Proof}: Suppose $e$ is an element that occurs more than $m/k$ times in the stream.
An occurance of $e$ is not counted in two different situations:

\begin{itemize}
	\item The dictionary is full and another element is added, the count of $e$ will be
decremented.
	\item Try to insert $e$ but dictionary is already full.
\end{itemize}

These events can only happen $m/k$ times.

}}

\vspace{2mm}\noindent\fbox{\parbox{\textwidth}{\textbf{Lemma 4:} If an element's count value is
more than $m/2$ in the dictionary, it is the majority element.
\newline

\textit{Proof}: From Leamm 3 we know that the count value is never greater than the true frequency
of an item. 

Trivially this means that if the count is greater than $m/2$ that element is the majority element.

}}

\vspace{2mm}\noindent\fbox{\parbox{\textwidth}{\textbf{Lemma 5:} The algorithm uses at most
$100 (log(n) + log(m))$ bits of memory.
\newline

\textit{Proof}: The algorithm will use $log(n) + log(m)$ bits of memory for each entry in the 
dictionary. $log(n)$ because the $key$ will always be in the span $1 \le key \le n$.
$log(m)$ because the $count$ will always be in the span $1 \le count \le m$.
The dictionary contains at most 100 entries and will therefore never use more than
$100 * (log(n) + log(m))$ bits of memory.

}}

\vspace{2mm}\noindent\fbox{\parbox{\textwidth}{\textbf{Lemma 6:} The algorithm uses less than
$\mathcal{O}(poly log(n + m))$ bits of memory.
\newline

\textit{Proof}: From Lemma 5 we know that the algorithm uses at most $100(log(n) + log(m))$ bits
of memory. The memory use will essentially increase linearly up until 100 unique elements exists
in the stream and after that it will be constant. So the worst case is  when $n = 100$ and $m \ge 100$.

\[100(log(100) + log(100)) = 400\]

\[\mathcal{O}(poly log(100 + 100)) = 2.30^x\]

If we choose $x$ to be large enough the algorithm will always use less memory.

}}

\end{document}