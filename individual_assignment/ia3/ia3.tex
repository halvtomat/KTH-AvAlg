\documentclass[11pt, a4paper]{article}

\usepackage{amsmath}

\newtheorem{theorem}{Theorem}[section]
\newtheorem{lemma}[theorem]{Lemma}

\title{Individual assignment 3, Problem 4 (Grade E)}
\author{Daniel Gustafsson}
\date{December 2021}

\begin{document}
\maketitle

\section*{The Probability}

\[3 * {1/6}^6 = 6.43 * 10^{-5} \]

\section*{The Proof}

\vspace{2mm}\noindent\fbox{\parbox{\textwidth}{\textbf{Lemma 1:} There are 3 and only 3 paths with
length 6 in the graph. The paths are the following: 

\vspace{1mm}\noindent \textbf{1} [ f c a d b e ]

\vspace{1mm}\noindent \textbf{2} [ f c a d g e ]

\vspace{1mm}\noindent \textbf{3} [ g f c a d b ]

\vspace{2mm}\textit{Proof}: Since this graph is small and with a limited number of paths to follow,
it is easy to manually follow the paths and find that the above paths are the only viable ones.

To test all paths in the graph, I try to start a path at each vertex and follow it until either a
path with lenght 6 is found or the path ends.

\vspace{2mm}\noindent
Starting from \textbf{a} there are 4 different paths to follow with lenghts 4, 4, 4 and 5.

\vspace{1mm}\noindent [ a d b e ]

\vspace{1mm}\noindent [ a d f c ]

\vspace{1mm}\noindent [ a d g e ]

\vspace{1mm}\noindent [ a d g f c ]

\vspace{2mm}\noindent
Starting from \textbf{b} there is only 1 path with lenght 2 to follow.

\vspace{1mm}\noindent [ b e ]

\vspace{2mm}\noindent
Starting from \textbf{c} there are 4 different paths to follow with lenghts 4, 5, 5 and 5.

\vspace{1mm}\noindent [ c a d f ]

\vspace{1mm}\noindent [ c a d b e ]

\vspace{1mm}\noindent [ c a d g e ]

\vspace{1mm}\noindent [ c a d g f ]

\vspace{2mm}\noindent
Starting from \textbf{d} there are 4 different paths to follow with lenghts 3, 3, 4 and 5.

\vspace{1mm}\noindent [ d b e ]

\vspace{1mm}\noindent [ d g e ]

\vspace{1mm}\noindent [ d f c a ]

\vspace{1mm}\noindent [ d g f c a ]

\vspace{2mm}\noindent
Starting from \textbf{e} there is only 1 path with length 1 to follow.

\vspace{1mm}\noindent [ e ]

\vspace{2mm}\noindent
Starting from \textbf{f} there are 2 different paths to follow with lenghts 6 and 6.

\vspace{1mm}\noindent [ f c a d b e ] (Path \textbf{1})

\vspace{1mm}\noindent [ f c a d g e ] (Path \textbf{2})

\vspace{2mm}\noindent
Starting from \textbf{g} there are 2 different paths to follow with lengths 2 and 6.

\vspace{1mm}\noindent [ g e ]

\vspace{1mm}\noindent [ g f c a d b ] (Path \textbf{3})

}}

\vspace{2mm}\noindent\fbox{\parbox{\textwidth}{\textbf{Lemma 2:} The probability for the Color Coding
algorithm to find a specific path of length 6 in the graph is ${1/6}^6$.

\vspace{2mm}\textit{Proof}: If the i$^{th}$ node of the path P is colored i, for all i, then the path will be found.
The probability for each node to be a specific color is $1/6$. The probability for 6 specific vertices to
each be a specific color is ${1/6}^6$. 

\vspace{2mm}\noindent
This is the same probability given as the lower bound of success in Chapter 3.2 of the course literature. 

}}

\vspace{2mm}\noindent\fbox{\parbox{\textwidth}{\textbf{Lemma 3:} The probability for the Color Coding
algorithm to find any path of length 6 in the graph is $3 * {1/6}^6$.

\vspace{2mm}\textit{Proof}: Combining the knowledge from Lemma 1 and Lemma 2 we get that the total probability
for the algorithm to find a path of lenght 6 is $3 * {1/6}^6$.

}}

\end{document}